\documentclass{scrartcl}

%\usepackage{fancyhdr}

\newcommand{\points}[1]
{\begin{flushright}\textbf{#1}\end{flushright}}
\newcommand{\sol}
{\textbf{L\"osung}:}

\usepackage{Sweave}
\begin{document}
\Sconcordance{concordance:zl_hs_2016_w13_sol11.tex:zl_hs_2016_w13_sol11.Rnw:%
1 9 1 1 0 1 1 1 47 134 1}



\thispagestyle{empty}

\titlehead
{
	ETH Z\"urich\\%
	D-USYS\\%
	Institut f\"ur Agrarwissenschaften
}

\title{\vspace{5ex} L\"osungen \"Ubung 11\\
       Probepr\"ufung \\
       Z\"uchtungslehre\\
       HS 2016 \vspace{3ex}}
\author{Peter von Rohr \vspace{3ex}}
\date{
  \begin{tabular}{lr}
  \textsc{Datum}  & \textsc{\emph{16. Dezember 2016}} \\
  \textsc{Beginn} & \textsc{\emph{09:15 Uhr}}\\
  \textsc{Ende}  & \textsc{\emph{ca 11:15 Uhr}}\vspace{3ex}
\end{tabular}}
\maketitle

% Table with Name
\begin{tabular}{p{3cm}p{6cm}}
Name:     &  \\
         &  \\
Legi-Nr:  & \\
\end{tabular}

% Table with Points

\vspace{3ex}
\begin{center}
\begin{tabular}{|p{3cm}|c|c|}
\hline
Aufgabe  &  Maximale Punktzahl     &  Erreichte Punktzahl\\
\hline
1        &  2  & \\
\hline
2        &  0  & \\
\hline
3        &  0  & \\
\hline
4        & 0   & \\
\hline
5        & 0   & \\
\hline
6        & 0   & \\
\hline
Total    &  2 & \\
\hline
\end{tabular}
\end{center}

\clearpage
\pagebreak

\section*{Aufgabe 1: Tierzucht (2)}
\begin{enumerate}
\item[a)] Nennen Sie die zwei Werkzeuge, welche in der Tierzucht bei der Auswahl potentieller Elterntiere verwendet werden.
\points{2}
\end{enumerate}

\sol

\begin{enumerate}
\item Selektion
\item gezielte Paarung
\end{enumerate}

\clearpage
\pagebreak

\begin{enumerate}
\item[b)] Die Vermehrung in Wildpopulationen verl\"auft etwas anders als in Nutztierpopulationen.
\points{2}
\end{enumerate}

\begin{center}
\begin{tabular}{|p{3cm}|p{5cm}|p{5cm}|}
\hline
  &  Wildpopulation  &  Nutztierpopulation \\
\hline
  &                  & \\
\hline
\end{tabular}
\end{center}

\sol

\begin{center}
\begin{tabular}{|p{3cm}|p{5cm}|p{5cm}|}
\hline
  &  Wildpopulation         &  Nutztierpopulation \\
\hline
  & nat\"urliche Selektion  &  k\"unstliche Selektion \\
\hline
  & zufällige Paarungen     &  gezielte Paarung \\
\hline
  & Vermehrung              &  gerichtete Selektion \\
\hline
  & optimale Anpassung an Umwelt  &  Optimierung des vom Menschen geschaffenen Zuchtsystems
\end{tabular}
\end{center}

\clearpage
\pagebreak


\section*{Aufgabe 2:  (0)}

\clearpage
\pagebreak

\section*{Aufgabe 3:  (0)}

\clearpage
\pagebreak

\section*{Aufgabe 4:  (0)}

\clearpage
\pagebreak

\section*{Aufgabe 5:  (0)}

\clearpage
\pagebreak

\section*{Aufgabe 6:  (0)}


\end{document}
